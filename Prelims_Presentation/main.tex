\documentclass{beamer}

%%% -------------- CREATE HANDOUTS -----------------------------------
%\documentclass[12pt, handout]{beamer}
%\usepackage{pgfpages}
%\pgfpagesuselayout{4 on 1}[letterpaper, landscape, border shrink=5mm]
%%% ------------------------------------------------------------------

\usepackage{ragged2e}
\usepackage[english]{babel}
\usepackage[utf8]{inputenc}
\usepackage{appendixnumberbeamer}
\usepackage{natbib}
\usepackage{textpos}
\usepackage{lipsum}
\usepackage{tikz}
\usepackage[percent]{overpic}
\usepackage{textcomp}
\usepackage{booktabs}
\usepackage{mathabx}
\usepackage{pifont}
\usepackage{bbding}
\usepackage{fontenc}

\mode<presentation>
  \usepackage{ru}
  %\usetheme{Warsaw}
  %\usecolortheme{seahorse}
  %\usefonttheme{default}
  \setbeamertemplate{caption}[numbered]
  %\setbeamertemplate{headline}{}
  %\setbeamertemplate{navigation symbols}{}
  \setbeamertemplate{bibliography item}[text]
\newcommand*\oldmacro{}%
	\let\oldmacro\insertshorttitle%
%\renewcommand*\insertshorttitle{%
%   \oldmacro\hfill%
%   \insertframenumber\,/\,\inserttotalframenumber}
\setbeamertemplate{section page}{
    \begin{centering}
    \vspace{1cm}
    \begin{beamercolorbox}[rounded=true,shadow=false,sep=4pt,center]{part title}
    \usebeamerfont{section title}\LARGE{\insertsection}\par
    \end{beamercolorbox}
\end{centering}}

%Modify Title page
\title[Decontamination Factors for Nuclear Forensics]{Experimental Characterization of Pu Separation by PUREX Process on a Low-Burnup, Pseudo-Fast-Neutron Irradiated DUO\tsbs{2} for Product Decontamination Factors and Nuclear Forensics}
\institute{Ph.D. Preliminary Examination}
\author{Paul Mendoza}
\date{02/27/2017}

\setbeamertemplate{title page}{
  \begin{centering}
    \vspace{-0.7cm}
    \begin{beamercolorbox}[rounded=true,shadow=false,sep=4pt,center]{part title}
      \usebeamerfont{title}\normalsize{\inserttitle}\par
    \end{beamercolorbox}
    \vspace{0.6cm}
    \begin{tabular}{rl}
    A PhD. Prelims Defense by:      & Paul Mendoza \\[1ex]
    Chair: & Dr. Sunil Chirayath\\[1ex]
    Committee Members: & Dr. Sean McDeavitt\\
    & Dr. Craig Marianno\\
    & Dr. Cody Folden III.
    \end{tabular}
    \begin{center}
      Monday, February 27, 2017, 10:00 am\\
      AIEN 304
    \end{center}
  \end{centering}
}



%\let\Sun\undefined   %to undefine something
%\usepackage{marvosym}
%\usepackage{scalerel}
%% \usepackage[
%%   style=numeric,
%%   citestyle=authoryeartitle
%% ]{biblatex}
\setbeamerfont{caption}{size=\tiny}

\setbeameroption{show notes}
\setbeamertemplate{note page}[plain]
\newcommand{\tss}{\textsuperscript}
\newcommand{\tsbs}{\textsubscript}

%Change Bullets in latex list
\setbeamertemplate{itemize item}{\scriptsize\raise1.25pt\hbox{\donotcoloroutermaths\ding{118}}}
\setbeamertemplate{itemize subitem}{\scriptsize\raise1.25pt\hbox{\ding{226}}}
\setbeamertemplate{itemize subsubitem}{\tiny\raise1.25pt\hbox{\ding{169}}}
\setbeamertemplate{enumerate item}{\insertenumlabel.}


%%TO SELECT SPECIFIC FONT
%{\FONTSIZE{2.5}{4}\SELECTFONT TOBESIZE}

%BIOLA SEAL
%\SETBEAMERTEMPLATE{BACKGROUND}{\TIKZ[OVERLAY, REMEMBER PICTURE]\NODE[XSHIFT=-2.3CM, YSHIFT=1.50CM, OPACITY=0.4]AT (CURRENT PAGE.SOUTH EAST){\INCLUDEGRAPHICS[WIDTH=4CM]{IMAGES}};}

\begin{document}

\setbeamertemplate{caption}{\raggedright\insertcaption\par}
\begin{frame}
	%% background
  \tikz[overlay, remember picture]\node[xshift=-3.5cm, yshift=3cm, opacity=0.1]at (current page.south east){\includegraphics[width=8cm]{figures/tamu_system_proposed_seal_042915}};
	%% Left-hand logo
  \begin{tikzpicture}[remember picture, overlay]
  \node [xshift = 3 cm, yshift=1.2cm] at (current page.south west){\includegraphics[width=5cm]{figures/tees_logo_primary_maroon}};
  \end{tikzpicture}
    %% Right-hand logo
  \begin{tikzpicture}[remember picture, overlay]
  \node [xshift = -3cm, yshift=1.2cm] at (current page.south east){\includegraphics[width=5cm]{figures/TEES_NSSPI_logo_HMaroon}};
  \end{tikzpicture}
  %% Upper logo right
    \begin{tikzpicture}[remember picture,overlay]
    \node[anchor=north east,yshift=2pt] at (current page.north east) {\includegraphics[height=0.8cm]{figures/NUENlogo}};
    \end{tikzpicture}
  %% Upper logo left
    %% \begin{tikzpicture}[remember picture,overlay]
    %% \node[anchor=north west,yshift=2pt] at (current page.north west) {\includegraphics[height=0.8cm]{NUENlogo}};
    %% \end{tikzpicture}

    \titlepage
    %\vspace{-1.8cm}
    %\begin{center}
    %  Presented at All Hands Meeting
    %\end{center}
\end{frame}

%Add Biola Seal
\setbeamertemplate{background}{\tikz[overlay, remember picture]\node[xshift=-2.5cm, yshift=2.5cm, opacity=0.05]at (current page.south east){\includegraphics[width=6cm]{figures/imageedit_2_7317234434}};}

%Add NSSPI to upper right
\addtobeamertemplate{frametitle}{}{%
  \begin{tikzpicture}[remember picture,overlay]
    \node[anchor=north east,yshift=2pt] at (current page.north east) {\includegraphics[height=0.8cm]{figures/NUENlogo}};
  \end{tikzpicture}
    \begin{tikzpicture}[remember picture,overlay]
      \node[anchor=north west,yshift=2pt] at (current page.north west) {\includegraphics[height=0.8cm]{figures/TEES_NSSPI_Acronym_logo_WHT}};
  \end{tikzpicture}
}


\begin{frame}{Outline}
\tableofcontents
\end{frame}

\section{Introduction}

\subsection{Motivation}
\begin{frame}{Motivation}
  \begin{itemize}
  \item{Current Events}
    \begin{itemize}
    \item{Joint Comprehensive Plan of Action}
    \item{Non-safeguarded reactors}
    \item{Islamic State of Iraq and Syria}
    \end{itemize}
  \item{Past Events}
    \begin{itemize}
    \item{Septemer 11, 2001}
    \end{itemize}
  \item{Limited scope of IAEA safeguards}
  \item{``the awful arithmetic of the atomic bomb''\tss{\cite{RN212}}}
  \item{Need for improved forensic capabilities\tss{\cite{RN103,RN98,RN113}}}
  \end{itemize}
\end{frame}

\note{Wanted to take some time to talk about the motivation behind this project.
  There are a lot of stuff going on in the world. Here I've listed a few that pertain to nuclear
  weapons.
  \begin{itemize}
  \item{Iran limiting capabilities for 8 years}
  \item{non-safeguarded reactors, in India for example}
  \item{North Korea, detonating a nuclear device October 2006}
  \item{ISIS, if they could, would probably want nuclear weapons, and they would use them on us}
  \end{itemize}
  As 911 indicated, we have enemies, who don't like us. To quote William Perry,
  ``Our greatest threat is a terrorist nuclear strike''. Lukily, aquiring nuclear weapons
  is no easy task. And thanks to organizations like the international atomic energy agency,
  or the treaty on the non proliferation of nuclear weapons, the international community is
  generally on board with nonproliferation. Nontheless, all sources of special nuclear material
  are not safeguarded, and the ``awful arithmetic of the atomic bomb'', to quote Eisenhower,
  doesn't leave much room for calculational errors.
  \begin{itemize}
  \item{Forensic capabilities need to improve, which we will talk about in a minute,
  but first, some definitions}
  \end{itemize}
}

\begin{frame}{Definitions}
  \begin{itemize}

  \end{itemize}
\end{frame}

\subsection{Background}
\begin{frame}{Forensic Context}
  \vspace{-1cm}
  \begin{center}
    ``The United States has developed a nuclear forensics capability that has been demonstrated
    in real-world incidents of \textbf{interdicted materials} and in exercises of actions
    required after a nuclear detonation. The committee, however, has concerns about the program
    and finds that without strong leadership, careful planning, and additional funds, these
    capabilities will decline''\tss{\cite{RN103}}
  \end{center}
  Major areas of concern include:
  \begin{itemize}
  \item{Organization}
  \item{Sustainability}
  \item{\textbf{Workforce and Infrastructure}}
  \item{\textbf{Procedures and Tools}}
  \end{itemize}
\end{frame}

\note{According to a report from the committee on nuclear forensics released in 2010,
    the United States forensic capability has been demonstrated in real world scenarios, but
    is at risk unless certain developmental requirements are met. The committee listed 4 areas
    of concern, where improvment is needed.
    \begin{itemize}
    \item{In terms of organization, nuclear forensics responsibility is shared by several agencies
      without central authority and with no consensus on strateic requirements to guide the program.}
    \item{For sustainability, our current capabilities are the fruit of the nuclear weapons program
      and our laboratory infrastructure, funding for both have been decling}
    \item{Skilled personnel in these areas are few, and key facilities are in need of replacement
      because they are old, outdated, or do not met modern environmental, health, or safety standards.}
    \item{In recent years, nuclear forensic techniques methodologies have been on the rise, some
      from presented from our own department, but according to this source, a large fraction
      of techniques are remnants of the cold war era, with less restrictions. Also, forensic
      excersises usually take months to complete, a time scale which is too long}
\end{itemize}}
\newpage
\note{Problem must be met on an administrative level, mentioned here because this proposed project
    will help, in a small way, with two areas of concern.}


\begin{frame}{Motivation}
  %\vspace{-1cm}
  \begin{itemize}
  \item{Weapons-grade Pu can be extracted from reactor discharged fuel
    with a burnup of about 1 (GWD/tU)}
  \item{Pu isotopes produced in irradiated fuel can vary} %depending on}
    %% \begin{itemize}
    %% \item{Burnup (irradiation history)}
    %% \item{Reactor neutron spectrum (core design)}
    %% \end{itemize}
  \item{Two examples of reactors which can intentionally
  discharge low burned fuel for extracting weapon-grade Pu are:}
  \begin{itemize}
  \item{Fast Breeder Reactor}
    \note[item]{Madras Atomic Power Station Kalpakkam, India}
    \note[item]{Expected criticality in Jan 2017}
    \note[item]{Cost from 450 million euros to 750 euros}
    \note[item]{Sodium-cooled reactor design - U238 for breeding}
    \note[item]{100 GWd/t for core, 40 year life, 1750 tonnes of sodium
      about 75\% of olympic sized swimming pool.}
    \note[item]{liquid sodium has a density a little less than water}
    \note[item]{MOX fuel (UO2 and PuO2) fuel}
    \note[item]{Fuel discharged at 100GWd/t, but I just mentioned
      that we are worried about 1GWd/t, mistake?}
  \item{CANDU Reactor
          \begin{figure}[H]
        \begin{flushright}
	   \includegraphics[scale = 0.2]{figures/nuclear-weapons-free-zone}
	\end{flushright}
      \end{figure}}
  \end{itemize}
%\item{In accord with the Indo-US 123 agreement, these reactors were not
%      required to be kept under IAEA safeguards}
  \end{itemize}
  %% \begin{tikzpicture}[remember picture, overlay]
  %%   \node [xshift = -3cm, yshift=1.2cm] at (current page.south east){\includegraphics[width=5cm]{nuclear-weapons-free-zone}};
  %% \end{tikzpicture}
\end{frame}

\begin{frame}{Smaller Picture}
  \begin{columns}
    \begin{column}{0.5\textwidth}
      \vspace{-10mm}
      \begin{itemize}
      \item{Attribution for unpurified Pu has been previously studied
        \tss{\cite{chirayath2015trace,scott2005nuclear,glaser2009isotopic}}}
      \item{Interdicted Pu would likely have been processed}
      \item{Lack of literature on decontamination factors and
        distribution coefficients for useful forensic elements (Cs, Sb,
        Eu, Rb, Sr, Nd, Pm, and Sm)} 
      \end{itemize}
    \end{column}
    \begin{column}{0.5\textwidth}
      \begin{figure}[H]
        \vspace*{-1cm}
        \begin{center}
	  \includegraphics[scale = 0.5]{figures/Stoller}
          \vspace{-0.5cm}
           \caption{\tiny{Adapted from Stoller\tss{\cite{stoller1961reactor}}}}
	\end{center}
      \end{figure}
    \end{column}
  \end{columns}  
\end{frame}


\section{Background}
\begin{frame}
\sectionpage
\end{frame}

\subsection{The PUREX Process}
\begin{frame}{What is PUREX - A type of laundry detergent?}
  \vspace{0.5cm}
  \begin{itemize}
  \item Plutonium Uranium Redox EXtraction 
    \begin{itemize}
    \item Liquid-liquid solvent extraction
    \item Many stages:
      \begin{enumerate}
      \item{Preparation for Dissolution}
      \item{Dissolution}
      \item{Preparation of Dissolved Feed}
      \item{Primary Decontamination - Extraction to
        organic\textsuperscript{\tiny{\AsteriskThin}}}
      \item{Scrubbing}
      \item{Plutonium Partition - Back-Extraction to
        aqueous\textsuperscript{\tiny{\AsteriskThin}}}
      \item{Plutonium Purification}
      \end{enumerate}
    \end{itemize}
  \end{itemize}
  \vspace{1.5cm}
  \textsuperscript{\tiny{\AsteriskThin\hspace{1mm}- Discussing Next}}
\end{frame}


\begin{frame}{Extraction}
  $UO_{2(aq)}^{2+}+2NO^{-}_{3(aq)}+2TBP_{(o)}
  \leftrightarrow UO_2(NO_3)_2\cdot2TBP_{(o)}$\tss{\cite{benedict1982nuclear}}
  $Pu^{4+}_{(aq)}+4NO^{-}_{3(aq)}+2TBP_{(o)}
  \leftrightarrow Pu(NO_3)_4\cdot 2TBP_{(o)}$
  \note[item]{Most of the fission products are
    left in the aqueous solution
    at valence III and V states\tss{\cite{kok2009nuclear}}}
  \vspace{-3mm}
  \begin{figure}[H]
    \vspace*{0.1cm}
    \begin{center}
      \includegraphics[scale = 0.5]{figures/Extraction}
    \end{center}
  \end{figure}
\end{frame}

\begin{frame}{Back-Extraction}
  $Pu(NO_3)_4(TBP)_{2(o)}+Fe^{2+}_{(aq)}\leftrightarrow Pu^{3+}_{(aq)}+4NO^{-}_{3(aq)}+2TBP_{(o)}$\tss{\cite{konings2006chemistry}}
  \note[item]{The fission products that contribute mostly
    to the radioactive contamination of product in PUREX
    are zirconium, niobium, and ruthenium - with multiple
    oxidation states.}
  \begin{figure}[H]
    \vspace*{0.1cm}
    \begin{center}
      \includegraphics[scale = 0.5]{figures/Back_Extraction}
    \end{center}
  \end{figure}
\end{frame}


%% \begin{frame}{Extraction and Back-extraction}
%%   \begin{columns}
%%     \begin{column}{0.5\textwidth}
%%       \vspace{-3mm}
%%       \begin{itemize}
%%       \item{Extraction}

%%       \item{Back-extraction}
%%         \begin{itemize}
%%           \item{An item}

%%         \end{itemize}
%%       \end{itemize}
%%     \end{column}
%%     \begin{column}{0.5\textwidth}

%%     \end{column}
%%   \end{columns}  
%% \end{frame}

\subsection{Distribution Coefficients}
\begin{frame}{Distribution Coefficients - The Missing link}
  \begin{itemize}
  \item{Distribution Coefficient (D): The ratio between the organic
  and aqueous phases (aka: D-values)}
    \begin{equation*}
      D=\frac{c_{o}}{c_{aq}}
    \end{equation*}
    \note[item]{Distribution coefficients can be reported
      in terms of volume basis (weight per unit volume), or a
      mass basis (mass of solute per unit mass of solute free
      solvent)- usually reported on volume basis}
  \item{Specific element to element}
  \item{Vary widely with:\tss{\cite{stoller1961reactor}}}
    \begin{itemize}
    \item{Composition of phases}
    \item{Solution saturation}
    \item{Temperature of the solvent}
    \end{itemize}
  \item{The fraction of mass, $f_o$ deposited in the organic
    phase, assuming a volume ratio between
    the aqueous and organic phases, $V_R$, is:}
    \begin{equation*}
      f_o=(1+D^{-1}V^{-1}_R)^{-1}
    \end{equation*}
    \note[item]{Note not a function of density, even though
      the two solutions have different densities, when solving for
      this value it cancels out}
    \note[item]{Solved this way to show, volume matters, and to give
      me a more intuitive sense of where things are going}
  \end{itemize}
\end{frame}

\subsection{Decontamination Factors}
\begin{frame}{Decontamination Factors - The Pot of gold}
  \begin{itemize}
  \item{After several cycles of Pu extraction/scrubbing/back-extraction
    are completed, the effectiveness of a PUREX cycle is described
    by the decontamination factor (DF):}
    \begin{equation*}
      DF_j=\frac{\left|\frac{c_j}{c_{Pu}}\right|_{initial}}
      {\left|\frac{c_j}{c_{Pu}}\right|_{final}}
    \end{equation*}
  \item{DFs are characteristic of different process cycles}
  \item{Larger values (10\tss{7}) for industrial scale PUREX (compared
  to benchtop)\tss{\cite{stoller1961reactor,benedict1982nuclear}}}
  \end{itemize}
\end{frame}

%% \begin{frame}{Complexity of reprocessing schemes}
%%   \begin{figure}[H]
%%     \vspace*{-1cm}
%%     \begin{center}
%%       \includegraphics[scale = 0.4]{figures/PUREX_Process}
%%       \caption{\tiny{Principal steps in PUREX\tss{\cite{benedict1982nuclear}}}}
%%     \end{center}
%%   \end{figure}
%%   \note[item]{Worry about dependancies of of DC
%%   Worry about non equilibrium
%%   flow rates, blah blah blah.
%%   Show mixer settler columns, show batch.
%%   What I want to do, get some distribution coefficients
%%   develop a reasonable process for isolating a large
%%   fraction of Pu, then, based on process, calculate what
%%   the decontamination factor should be, and then actually measure it}
%% \end{frame}


%% \begin{frame}{Complexity of reprocessing schemes}
%%   \begin{figure}[H]
%%     \vspace*{-1cm}
%%     \begin{center}
%%       \includegraphics[scale = 0.4]{figures/mixer}
%%       \caption{\tiny{Mixer-Settler Diagram}}
%%     \end{center}
%%   \end{figure}
%% \end{frame}


%% \begin{frame}{Complexity of reprocessing schemes}
%%   \begin{figure}[H]
%%     \vspace*{-1cm}
%%     \begin{center}
%%       \includegraphics[scale = 0.35]{figures/stage}
%%       \caption{\tiny{Multistage countercurrent solvent extraction.
%%           M, mixer; S, settler.\tss{\cite{benedict1982nuclear}}}}
%%     \end{center}
%%   \end{figure}
%% \end{frame}


%% \begin{frame}{Complexity of reprocessing schemes}
%%   \begin{figure}[H]
%%     \vspace*{-1cm}
%%     \begin{center}
%%       \includegraphics[scale = 0.4]{figures/column}
%%       \caption{\tiny{Pulsed Column Diagram}}
%%     \end{center}
%%   \end{figure}
%% \end{frame}

\section{Previous Work}
\begin{frame}
\sectionpage
\end{frame}

\subsection{Experiment}
\begin{frame}{Irradiation}
  \begin{columns}
    \begin{column}{0.5\textwidth}
      \vspace{-1cm}
      \begin{itemize}
      \item{12.9 $\pm$ 0.1 mg of DUO\tsbs{2} was irradiated}
        \begin{itemize}
        \item{High Flux Isotope Reactor at Oak Ridge National Laboratory}
        \end{itemize}
      \item{Burnup was 4.43 $\pm$ 0.31 GWd/tHM\tss{\cite{swinney2015experimental}}}
      \item{0.196 $\pm$ mg of total Pu was produced as measured by ICP-MS}
      \end{itemize}
    \end{column}
    \begin{column}{0.5\textwidth}
      \begin{figure}[H]
        \vspace*{-1cm}
        \begin{center}
	   \includegraphics[scale = 0.4]{figures/irradiated}
           %\caption{\tiny{Picture of irradiated sample}}
	\end{center}
      \end{figure}
    \end{column}
  \end{columns}  
  \end{frame}

\begin{frame}{Dissolution of the spent fuel pellet}
      \begin{figure}[H]
        \vspace*{-1cm}
        \begin{center}
	   \includegraphics[scale = 0.65]{figures/dissolution}
	\end{center}
      \end{figure}
\end{frame}

\begin{frame}{Glovebox}
      \begin{figure}[H]
        \vspace*{-1cm}
        \begin{center}
	   \includegraphics[scale = 0.5]{figures/glovebox}
	\end{center}
      \end{figure}
\end{frame}

\begin{frame}{Experiments}
  \begin{itemize}
  \item{Single stage extraction and back-extraction}
    \begin{itemize}
    \item{Purpose: quantify product recovery, D-values and DF values
      for single stage extraction and back extraction}
    \item{Conditions:}
    \end{itemize}
  \end{itemize}
  \begin{center}
    \vskip -0.2cm
    {\fontsize{2.5}{4}\selectfont
      \begin{tabular}{l  c  c}\toprule
        Starting Solution  & Extraction Solution
        & Back extraction solution \\ \midrule \vspace{0.1cm}
        4 M nitric acid & 30\% vol.\% TBP, 70 vol.\% kerosene & 0.024 M ferrous sulfamate in 0.75 M nitric acid \\ \bottomrule
      \end{tabular}
      }
  \end{center}
  \begin{itemize}
  \item{Multi-contact extraction and back-extraction}
    \begin{itemize}
    \item{Purpose: Maximize recovery of Pu with 4 extractions,
      3 back extractions}
    \item{Conditions:}
    \end{itemize}
  \end{itemize}
    \begin{center}
    \vskip -0.2cm
    {\fontsize{2.5}{4}\selectfont
      \begin{tabular}{l  c  c}\toprule
        Starting Solution  & Extraction Solution
        & Back extraction solution \\ \midrule \vspace{0.1cm}
        4 M nitric acid & 30\% vol.\% TBP, 70 vol.\% kerosene & 0.024 M ferrous sulfamate in 4 M nitric acid \\ \bottomrule
      \end{tabular}
      }
  \end{center}
\end{frame}

\subsection{Recovery of Pu and U}
\begin{frame}{Previous Experiment Results}
  \begin{block}{Recoveries of U and Pu}
    \begin{center}
      \vskip -0.2cm
   %{\fontsize{2.5}{4}\selectfont
  \begin{tabular}{l  c  c}\toprule
                & Pu Recovery & U Recovery \\ \midrule \vspace{0.1cm}
   Single stage         & (83.4$\pm$9.5)\% & (11.2$\pm$1.3)\% \\
   Multi-contact Cycle 1 & (99.7$\pm$4.2)\% & (6.8$\pm$0.3)\% \\
   Multi-contact Cycle 2 & (93.0$\pm$4.6)\% & (6.6$\pm$0.3)\% \\
   Overall Experiment 2 & (92.7$\pm$6.0)\% & (0.45$\pm$0.03)\% \\ \bottomrule
  \end{tabular}
  %}
  \end{center}
  \end{block}
\end{frame}

\subsection{Experimental Decontamination Factors}
\begin{frame}{Previous Experiment Results}
  \vspace{-0.6cm}
  \begin{block}{Decontamination Factors}
    \begin{center}
      \vskip -0.2cm
  {\fontsize{7}{11.2}\selectfont
  \begin{tabular}{l  c  c c c c}\toprule
   Element (Z)  & SS & Error & MC Cycle 1 & Error & Isotopes Used\\ \midrule 
   Rb(37) & 39.0 & 5.9 & 11.8 & 0.8 & $^{85}$Rb \\
   Sr(38) & 283  & 43  & 84.6 & 5.9 & $^{90}$Sr \\
   Mo(42) & 5.7  & 0.8 & 1.9  & 0.2 & $^{97,98,100}$Mo \\
   Ru(44) & 59.2 & 6.4 & 16.6 & 2.5 & $^{101,102,104}$Ru \\
   Pd(46) & 65   & 14  & 8.9  & 1.2 & $^{110}$Pd \\
   Cd(48) & 74   & 17  & 22.1 & 2.5 & $^{112}$Cd \\
   Cs(55) & 177  & 28  & 52.9 & 3.9 & $^{133}$Cs \\
   Ce(58) & 43   & 16  & 11.5 & 4.9 & $^{140,142}$Ce \\
   Nd(60) & 19.2 & 2.1 & 5.9  & 0.4 & $^{143}$Nd \\
   Pm(61) & 12.8 & 1.9 & 3.9  & 0.3 & $^{147}$Pm \\
   Sm(62) & 11.5 & 1.5 & 3.6  & 0.3 & $^{151}$Sm \\
   Eu(63) & 10.0 & 1.4 & 3.6  & 0.3 & $^{154}$Eu \\
   U(92) & 7.4   & 1.2 & 14.7 & 0.9 & $^{238}$U \\ \bottomrule
  \end{tabular}
  }
  \end{center}
  \end{block}
\end{frame}

\begin{frame}{Conclusions}
  \begin{itemize}
  \item{Two PUREX experiments were conducted}
    \begin{itemize}
    \item{Single stage: Determined DC values for Pu, U and several FP}
    \item{Multi-contact: Utilized Experiment 1 to recover over 92\%
      of Pu while leaving less than 1\% of the U}
    \end{itemize}
  \item{DF values were measured for 12 FP elements}
  \item{DF values were lower than those typically found in industrial
    scale PUREX plants due to multiple extraction and back-extraction
    steps without an intermittent scrubbing step.}
  \item{This work provide DF data that will be built upon for
        nuclear forensic investigations of interdicted Pu.}
  \end{itemize}
\end{frame}


\section{Future Work}
\begin{frame}
\sectionpage
\end{frame}

\begin{frame}{Future Work}
  \begin{itemize}
  \item{Modify Multi-contact extraction, to recover a larger
    fraction of Pu}
  \item{Investigation of how D-values for (Cs, Sb,
    Eu, Rb, Sr, Nd, Pm, and Sm) change as a function of
    nitric acid concentration}
  \item{Determine statistical uncertainty of D and DF values.}
    \begin{itemize}
    \item{Repeat above experiments 3-5 times}
    \end{itemize}
  \item{Connect D-values with process information to DF values}
\end{itemize}
\end{frame}

\appendix
\section{Questions?}
\begin{frame}
\sectionpage
\end{frame}

\begin{frame}{Previous Experiment Results}
  \begin{figure}[H]
    \vspace*{-.1cm}
    \begin{center}
      \includegraphics[scale = 0.6]{figures/df}
      \vspace{-0.5cm}
      \caption{\tiny{Decontamination Factors for multi-contact extraction.}}
    \end{center}
  \end{figure}
\end{frame}

\begin{frame}[allowframebreaks]{References}
\def\newblock{}
\nocite{*}
%\scriptsize{\bibliographystyle{plain}}
\scriptsize{\bibliographystyle{unsrt}}
\bibliography{references}
\end{frame}

\begin{frame}{Mass Spec}
  \begin{figure}[H]
    \vspace*{-1cm}
    \begin{center}
      \includegraphics[scale = 0.75]{figures/instrument_response}
    \end{center}
  \end{figure}
\end{frame}

\end{document}
